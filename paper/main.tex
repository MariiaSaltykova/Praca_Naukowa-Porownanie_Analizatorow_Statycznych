\documentclass[a4paper,12pt]{article}
\usepackage[utf8]{inputenc}
\usepackage{graphicx}
\usepackage{hyperref}
\usepackage{amsmath, amssymb}
\usepackage{natbib}
\usepackage{geometry}
\geometry{a4paper, margin=1in}
\usepackage[T1]{fontenc}
\usepackage[polish]{babel}
\usepackage[utf8]{inputenc}
\usepackage{abstract} % Для улучшения оформления аннотации
\usepackage{titlesec} % Для настройки заголовков
\usepackage{setspace} % Для настройки межстрочного интервала

% Настройка заголовков
\titleformat{\section}{\normalfont\Large\bfseries}{\thesection}{1em}{}
\titleformat{\subsection}{\normalfont\large\bfseries}{\thesubsection}{1em}{}
\titleformat{\subsubsection}{\normalfont\normalsize\bfseries}{\thesubsubsection}{1em}{}

% Настройка межстрочного интервала
\onehalfspacing

\title{Praca naukowa dotycząca porównania istniejących analizatorów statycznych}
\author{Maryia Babinskaya, Alicja Bieżychudek, Mariia Saltykova \\ Vladyslav Khabanets, Tomasz Kosmulski, Maksim Zdobnikau \\ Uniwersytet Jagielloński}

\date{\today}

\begin{document}
\maketitle

\begin{abstract}
\noindent
\textbf{Tło:} Analizatory statyczne odgrywają kluczową rolę w zapewnieniu jakości oprogramowania, wykrywając potencjalne błędy i luki w kodzie źródłowym. Pomimo ich szerokiego zastosowania, istnieje potrzeba systematycznego porównania ich skuteczności, wydajności i funkcjonalności.\\
\textbf{Cel:} Celem niniejszej pracy jest porównanie istniejących analizatorów statycznych pod kątem ich możliwości, ograniczeń i zastosowań w różnych kontekstach programistycznych.\\
\textbf{Metody:} \\
\textbf{Wyniki:} \\
\textbf{Wnioski:} \\
\vspace{0.5cm}
\noindent
\textbf{Słowa kluczowe:} analizatory statyczne, jakość oprogramowania
\end{abstract}

\section{Wprowadzenie}
Analizatory statyczne są nieodzownym narzędziem w procesie tworzenia oprogramowania, umożliwiając wykrywanie błędów na wczesnych etapach rozwoju. Pomimo ich szerokiego zastosowania, istnieje wiele wyzwań związanych z ich skutecznością i wydajnością. Niniejsza praca ma na celu porównanie istniejących analizatorów statycznych, aby pomóc deweloperom w wyborze najbardziej odpowiedniego narzędzia dla ich potrzeb.

\section{Powiązane prace}

\section{Metody}
W niniejszej pracy przeprowadzono przegląd literatury oraz eksperymenty porównawcze. Wybrano pięć popularnych analizatorów statycznych: A, B, C, D i E. Każde narzędzie zostało przetestowane na zbiorze testowym zawierającym różne typy błędów, w tym błędy pamięci, wycieki zasobów i luki bezpieczeństwa. Skuteczność i wydajność każdego narzędzia została oceniona na podstawie liczby wykrytych błędów oraz czasu wykonania analizy.

\section{Analiza wyników}


\section{Dyskusja}
Wyniki sugerują, że wybór analizatora statycznego powinien być uzależniony od konkretnych potrzeb projektowych.

\section{Wnioski i dalsze badania}


\section*{Podziękowania}
Autorzy pragną podziękować Uniwersytetowi Jagiellońskiemu za wsparcie w realizacji niniejszego projektu.

\section*{Oświadczenie o wkładzie autorów}
Maryia Babinskaya: \\
Alicja Bieżychudek: \\
Mariia Saltykova: \\
Vladyslav Khabanets: \\
Tomasz Kosmulski: \\
Maksim Zdobnikau: \\

\section*{Oświadczenie o konflikcie interesów}
Autorzy deklarują brak konfliktu interesów.

\section*{Dostępność danych}
Dane użyte w badaniu są dostępne na platformie \href{https://example.com}{example.com}.


\bibliographystyle{plain}
\bibliography{references} % Dodaj plik references.bib do katalogu projektu

\end{document}